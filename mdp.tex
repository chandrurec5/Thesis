%!TEX root =  autocontgrlp.tex
\chapter{Markov Decision Processes (MDPs)}
In this section, we briefly discuss the basics of Markov Decision Processes (MDPs) (the reader is referred to \cite{BertB,Puter} for a detailed treatment).\\
An MDP is a $4$-tuple $\langle S,A,P,g\rangle$, where $S$ is the state space, $A$ is the action space, $P$ is the probability transition kernel and $g$ is the reward function. We consider MDPs with large but finite number of states. We also assume that the number of actions is finite.
Without the loss of generality, we set  $S=\{1,2,\ldots,n\}$  and the action set is given by $A=\{1,2,\ldots,d\}$. For simplicity, we assume that all actions are feasible in all states. The probability transition kernel $P$ specifies the probability $p_a(s,s')$ of transitioning from state $s$ to state $s'$ under the action $a$. We denote the reward obtained for performing action $a\in A$ in state $s\in S$ by $g_a(s)$.

By a policy, we mean a sequence $\pi=\{\pi_0,\ldots,\pi_n,\ldots\}$ of functions $\pi_i, i\geq 0$ that describe the manner in which an action is picked in a given state at time $i$. The most general class of policies is the class of \emph{history-dependent randomized policies} denoted by $\pi$ and is defined below.
\begin{definition}
Let $h_n=\{s_0,\ldots,s_n, \pi_0, \ldots, \pi_{n-1}\}$ denote the \emph{history}. A policy $\pi$ is said to be a \emph{history-dependent randomized} policy if $\pi_n,n\geq 0$ is such that for every $s\in S$, $\pi_n(s,\cdot|h_n)$ is a probability distribution over the set of actions.
\end{definition}
We denote the class of policies by $\Pi$. The infinite horizon discounted reward corresponding to state $s$ under a policy $\pi$ is denoted by $J_\pi(s)$ and is defined by
\begin{align}
J_\pi(s)\stackrel{\Delta}{=}\E[\sum_{n=0}^\infty \alpha^n g_{a_n}(s_n)|\pi],\nn
\end{align}
where $a_{n} \sim \pi_n(s,\cdot)$ and $\alpha \in (0,1)$ is a given discount factor.  Here $J_\pi(s)$ is known as the value of the state $s$ under the policy $\pi$, and the vector quantity $J_\pi\stackrel{\Delta}{=}(J_\pi(s))_{s\in S}\in R^n$ is called the value-function corresponding to the policy $\pi$. The optimal value function $J^*$ is defined as $J^*(s)\stackrel{def}{=}\underset{\pi \in \Pi}{\max} J_\pi(s)$.\par
A stationary deterministic policy (SDP) is one where $\pi_n\equiv u$ for all $n\geq 0$ for some $u\colon S \ra A$. By abuse of notation we denote the SDP by $u$ itself instead of $\pi$. In the setting that we consider, one can find an SDP that is optimal \cite{BertB,Puter}, i.e., there exists an SDP $u^*$ such that $J_{u^*}(s)=J^*(s),\forall s\in S$. In this paper, we restrict our focus to the class $U$ of SDPs and without loss of generality we assume that there exists a unique SDP $u^*$ that is optimal. Under a stationary policy $u$ (or $\pi$), the MDP is a Markov chain and we denote its probability transition kernel by $P_u=(p_{u(i)}(i,j),i,j=1,\ldots,n)$ (or $P_\pi=(p_{\pi(i)}(i,j),i,j=1,\ldots,n)$, where $p_{\pi(i)}(i,j)=\sum_{a\in A}\pi(i,a)p_a(i,j)$ and $\pi(i)=(\pi(i,a), a\in A)$).\par
Given an MDP, our aim is to find the optimal value function $J^*$ and the optimal policy SDP $u^*$. The optimal policy and value function obey the Bellman equation (BE): for all $ s \in S$,
\begin{subequations}\label{bell}
\begin{align}
\label{bellval}J^*(s)&=\max_{ a\in A}\big(g_a(s)+\alpha \sum_{s'}p_a(s,s')J^*(s')\big),\\
\label{bellpol}u^*(s)&=\arg\max_{ a\in A}\big(g_a(s)+\alpha \sum_{s'}p_a(s,s')J^*(s')\big).
\end{align}
\end{subequations}
If $J^*$ is computed first, then $u^*$ can be obtained by substituting $J^*$ in \eqref{bellpol}. The Bellman operator $T$ is defined using the model parameters of the MDP as follows:
\begin{definition}
The Bellman operator $T: \R^n \to \R^n$ is defined by 
\begin{align}
(TJ)(s)=\max_{a \in A}\big(g_a(s)+\alpha \sum_{s'} p_a(s,s')J(s')\big)\,, 
\end{align}
where $J\in \R^n$ and $J_s = J(s)$, $s\in S = \{1,\dots,n\}$.  
Similarly one can define the Bellman operator $T_u:\R^n \to \R^n$ restricted to an SDP $u$ as follows:
\begin{align}
(T_uJ)(s)=g_{u(s)}(s)+\alpha \sum_{s'} p_{u(s)}(s,s')J(s')\,.
\end{align}
\end{definition}
Given $J \in \R^n$, $TJ$ is the `one-step' greedy value function. It is also useful to define the notion of a one-step greedy policy as below:
\begin{definition}
A policy $\tilde{u}$ is said to be greedy with respect to $\tilde{J}$ if
\begin{align}\label{subpol}
\tilde{u}(s)=\underset{a \in A}{\arg\max}  \big(g_a(s)+\alpha\sum_{s'} p_a(s,s')\tilde{J}(s')\big)
\end{align}
\end{definition}
We also define the Bellman operator $H$ for action values  \cite{BertB}:
\begin{definition}\label{bellactval}
Let $H: \R^n \to \R^{nd}$ be defined as follows: For $J\in \R^n$,
\begin{align}
HJ&=\left[ {\begin{array}{c} H_1 J  \\ \vdots \\ H_d J\end{array}} \right]\in \R^{nd},\mbox{where}\nn\\
(H_a J)(s)&= g_a(s)+\alpha \sum_{s'}p_a(s,s') J(s'), \qquad s\in S, a\in A.
\end{align}
\end{definition}
 We now state without proof the important properties related to the Bellman operator. Though in these results, we make use of the Bellman operator $T$, the results also trivially hold for $T_u$ as well.
\subsection{Properties of $T$}
\begin{lemma}\label{maxnorm}
$T$ is a $\max$-norm contraction operator, i.e., given $J_1, J_2 \in \mathbf{R}^n$,
\begin{align}
||TJ_1-TJ_2||_\infty\leq \alpha ||J_1-J_2||_\infty.
\end{align}
\end{lemma}
\begin{corollary}\label{uniquesol}
$J^*$ is a unique fixed point of $T$, i.e., $J^*=TJ^*$.
\end{corollary}
The Bellman operator $T$ exhibits two more important properties presented in the following lemmas (see \cite{BertB} for proofs):
\begin{lemma}\label{monotone}
$T$ is a monotone map, i.e., given $J_1,J_2 \in \R^n$ such that $J_2\geq J_1$, we have $T J_2\geq T J_1$. Further, if $J\in \R^n$ is such that $J\geq TJ$, it follows that $J\geq J^*$.
\end{lemma}
\begin{lemma}\label{shift}
Given $J\in \R^n$, $t \in \R$ and $\mathbf{1} \in \R^n$, a vector with all entries $1$, we have
\begin{align}
T(J+t\mathbf{1})=TJ+\alpha t\mathbf{1}.
\end{align}
\end{lemma}
Note that Lemmas~\ref{maxnorm}-\ref{shift} also hold for the Bellman operator $H$ defined for the action values in Definition~\ref{bellactval}.\par
Solving an MDP involves handling two sub-problems namely the problem of \emph{control} and the problem of \emph{prediction}. The problem of \emph{control} deals with coming up with a good (and if possible the optimal) policy. Often, in order to solve the problem of \emph{control}, one needs to solve the problem of \emph{prediction}, which deals with computing the value function $J_u$ of the policy $u$. The fact that the two problems are related is reflected in the Bellman equation in \eqref{bell}, where $J^*$ from \eqref{bellval} is used in \eqref{bellpol} to obtain the optimal policy $u^*$. Thus, the Bellman equation is at the heart of the solution methods to MDPs. Any solution method to MDP is said to be complete only if it satisfactorily (with provable performance guarantees) addresses both the prediction and the control problems. \\
The basic solution methods namely value iteration, policy iteration and linear programming (LP) formulation \cite{BertB} solve both the control and prediction problems. Of the three basic methods, in this paper, we are interested in the LP formulation given by
\begin{align}\label{mdplp}
\begin{split}
\min_{J\in \R^n}\, &c^\top J\\
\text{s.t}\mb &J(s)\geq g_a(s)+\alpha\sum_{s'}p_a(s,s')J(s'), \qquad s\in S, a \in A,
\end{split}
\end{align}
where $c\in \R^n_+$ is any vector whose components are all non-negative. One can show that $J^*$ is the solution to the LP formulation \eqref{mdplp} \cite{BertB}. 
Also, of the three methods, value iteration and LP formulation are value function based methods, i.e., they compute $J^*$ directly and then $u^*$ is obtained by plugging $J^*$ in \eqref{bellpol}.\\
While the basic methods (i.e., VI, PI and LP) can be used to compute exact values $J^*$ and $u^*$ for MDPs with a small number of states, they are computationally expensive in the case of MDPs with a large number of states.\\
