\section{Introduction}
A typical stochastic approximation \cite{kushner} (SA) scheme can be specified as under:
\begin{align}\label{rmsa}
x_{n+1}=x_n+a(n)[h(x_n)+M_{n+1}], n\geq 0,
\end{align}
where the iterates $x_n \in \R^d$, $h\colon \R^d \ra \R^d$, and $M_{n+1}$ are zero-mean noise terms and $a(n)$ are diminishing step sizes. The earliest and most representative example of an SA scheme is the 
Robbins-Monro (RM) algorithm, an iterative procedure to compute the zeros of a function using only 
its noisy observations. Here, $h$ is the function whose unique zero say $x^* \in \R^d$ needs to be 
computed and at any point $x_n$, the value of the function $h(x_n)$ is corrupted by an additive noise 
$M_{n+1}$. Other instances of SA schemes are the stochastic gradient schemes wherein $h$ is the 
gradient of some other function $f\colon \R^d \ra \R$, i.e., $h(x)=\nabla f(x)$. Stochastic gradient 
schemes are found to be useful in actor-critic \cite{NAC,IAC,KAC,BAC,BSPSA,Pet-Sch,ALP-Bor,SB1} and policy gradient \cite{Bax-Bar,Amari,SMSM,Kak,SP} reinforcement learning algorithms as well as stochastic optimization algorithms \cite{SPSA,TTSPSA,SB-VB,Mar-Tsit,SB-VB2,SB3,BTS,BO,BADA}.\\
\indent The conditions under which the above scheme converges have been studied widely (\cite{kushner,SA}) 
in the literature. In general, analysis of SA schemes involves two important steps namely,
\begin{enumerate}
\item establishing $\emph{stability}$ of the iterates $x_n$, i.e., showing that 
$\sup_n \parallel x_n\parallel <\infty$ with probability one, and  \label{stb}
\item establishing $\emph{convergence}$ to a desired solution $x^*$, i.e., showing that 
$x_n \rightarrow x^*$ almost surely as $n \rightarrow \infty$. \label{conv}
\end{enumerate}
%Often \ref{stb} is required to show \ref{conv} (\cite{SA}).
In situations where stability of iterates cannot be established analytically, a common procedure 
is to pick a suitable compact subset of the parameter space onto which one would projects the iterates, 
thereby ensuring their boundedness, i.e., $\sup_n \parallel x_n\parallel <\infty$.\\
\indent The ODE method (\cite{SA,BMSTAB}) is a useful analytical tool to study the stability as well as 
convergence of the stochastic iterates. In the ODE method, one considers \eqref{rmsa} to be a 
$\emph{noisy-discretized}$ version of the ODE,
\begin{align}\label{odebasic}
\dot{x}(t)=h(x(t)).
\end{align}
An important notion associated with the ODE method is that of $\emph{timescale}$ that is
defined using the quantity $t(n)=\sum_{m=0}^{n-1} a(m),\mb n\geq 1$. Let $\phi(t,x), t\geq 0$ 
denote the trajectory of \eqref{odebasic} with initial condition $x(0)=x$. When $\sup_n\parallel
x_n\parallel <\infty$, 
the trajectory $\phi(t,x)$ sampled at instants $t(n)$, i.e., $\{\phi(t(n),x_0), n\geq 0\}$ approximates the 
iterates ${x_n}$ (Lemma~$1$, Chapter $2$,\cite{SA}). The SA scheme can then be shown to converge to a 
compact connected internally chain transitive invariant set of the ODE in \eqref{odebasic}. 
Thus the convergence of iterates in the SA scheme \eqref{rmsa} can be understood by looking at the 
asymptotic equilibria of the ODE in \eqref{odebasic}.\\
\indent
The ODE method can also be used to establish the stability of iterates $\{x_n\}$ which involves 
understanding the behavior of \eqref{rmsa} when $\parallel x_n\parallel$ gets larger. The idea is to observe
the evolution of the continuously interpolated trajectory obtained from the iterates $\{x_n\}$ every 
$T$ instants and then rescaling it to the unit ball if the trajectory goes out of it. 
The rescaled iterates $\{\tilde{x}_n\}$ then differ from the original iterates $\{x_n\}$ only by a 
constant factor. The rescaled iterates track the rescaled ODE,
\begin{align}
\dot{x}(t)=h_{c}(x(t)), \text{where}\mbox{ } h_c(x)\stackrel{def}{=}\frac{h(cx)}{c}, \forall c\geq 1.
\end{align}
As $c \ra \infty$, one obtains the following $\emph{limiting}$ ODE (assuming the limit exists):
\begin{align}\label{lode}
\dot{x}(t)=h_\infty(x(t)), \text{where} \mbox{ } h_\infty(x)=\lim_{c \ra \infty} h_c(x).
\end{align}
When the ODE \eqref{lode} is globally asymptotically stable to the origin, \cite{BMSTAB} shows that 
$\sup_n \parallel x_n\parallel <\infty$ almost surely. This is because outside a set of zero probability,
as $\parallel x_n\parallel$ grows larger, the scaling factor tends to $\infty$ and 
$\tilde{x}_n$ can be approximated 
by the trajectory of the limiting ODE \eqref{lode}. Since the ODE \eqref{lode} is stable to the origin, 
the scaled iterates $\{\tilde{x}_n\}$ fall back to the origin. The stability of $\{x_n\}$ then follows 
upon noting that these iterates differ from $\tilde{x}_n,n\geq 1$ only by a scaling factor.\\
\indent The SA scheme in \eqref{rmsa} is a $\emph{single}$ timescale stochastic approximation scheme. 
A two-timescale stochastic approximation scheme \cite{TTSAP,BADA,SB-VB2,TTSPSA,BO,BTS} 
is as under:
\begin{subequations}\label{ttsaintro}
\begin{align}
\label{frec} x_{n+1}&=x_n+a(n)[h(x_n,y_n)+M^{(1)}_{n+1}],\\
\label{srec} y_{n+1}&=y_n+b(n)[g(x_n,y_n)+M^{(2)}_{n+1}], 
\end{align}
\end{subequations}
where, $x_n \in \R^{d_1}, y_n \in \R^{d_2}$, $a(n)$ and $b(n)$ are both diminishing step-size schedules 
except that $\frac{b(n)}{a(n)} \rightarrow 0$ as $n\rightarrow \infty$. 
The two timescales here are, the $\emph{slower}$, corresponding to $t^s(n)=\sum^{n-1}_{m=0} b(m)$, 
and the $\emph{faster}$,  corresponding to  $t^f(n)=\sum^{n-1}_{m=0} a(m)$ respectively, on which 
the two recursions proceed. The terms $\emph{faster}$ and $\emph{slower}$ signify the rates of growth 
of $t^f(n)$ and $t^s(n)$ respectively. Many iterative schemes might involve calls to subroutines which 
themselves are iterative in nature. Thus the $\emph{outer}$-loop in such schemes must typically wait for 
the iterates in the $\emph{inner}$-loop to converge before proceeding to its next step. 
Instead, the same effect can also be achieved by a two-timescale recursion with outer and inner loop 
iterates on the slower and faster timescales respectively. The condition $\frac{b(n)}{a(n)} \ra 0$ 
ensures that the slower timescale iterates do not move much compared to the faster timescale iterates, 
thereby producing the effect equivalent to that of a $\emph{nested}$-loop. As a result,
multi-timescale stochastic approximation is commonly used in algorithms that involve estimation of 
$\emph{nested}$ quantities \cite{NAC,IAC,KAC,TTSAP,SB-VB2,BTS}. Study of two timescale SA schemes is 
also important as it is easy to extend the arguments made for two timescale recursions to those with 
even more timescales.\\
\indent Under the assumption that $\sup_n\parallel x_n\parallel <\infty$, $\sup_n\parallel y_n\parallel
<\infty$, the ODE method can 
be applied to establish the convergence of iterates in \eqref{ttsaintro}, see \cite{TTSAP} 
and Chapter~$6$, \cite{SA}. In this paper, we provide {\em the first} conditions in the literature
that imply stability of the iterates in \eqref{ttsaintro}. These are conditions that imply 
$\sup_n\parallel x_n\parallel <\infty$ and $\sup_n\parallel y_n\parallel
<\infty$ with probability one. Stability results 
for the single timescale recursions, see \cite{BMSTAB} and Chapter $3$ of \cite{SA}, cannot be applied 
directly to the two-timescale scheme in \eqref{ttsaintro}, because the iterates $\{x_n\}$ and $\{y_n\}$ 
are coupled and they evolve along separate timescales. Nonetheless, while our analysis resembles 
\cite{SA,TTSAP} 
in that we study scaled ODEs, it differs considerably from \cite{SA} in that we study two separate 
sets of scaled ODEs for the two different timescales. Since the iterates $x_n$ evolve on the faster 
timescale, it is natural to first characterize the growth of $\{x_n\}$ before studying the stability
of $\{y_n\}$. So, we first study the following scaled ODE and its limiting version:
\begin{align}
\label{hcf} \dot{x}(t)&={h}_c(x(t),y), \text{where}\mbox{ } {h}_c(x,y)\stackrel{def}{=}\frac{h(cx,cy)}{c}, 
c \geq 1,\\
\label{hif} \dot{x}(t)&=h_\infty(x(t),y), \text{where}\mbox{ } {h}_\infty(x,y)=\lim_{c \ra \infty}h_c(x,y).
\end{align}
Equations \eqref{hcf} and \eqref{hif} also allude to the fact that the slower timescale iterates 
appear to be constant, i.e., $y(t)\equiv y$ $\forall t$, when viewed from the faster timescale 
\cite{SA}.\\
Under the assumption that the limiting ODE \eqref{hif} has a unique globally asymptotically stable 
equilibrium $\lambda_\infty(y)$, where $\lambda_\infty \colon \R^{d_2} \ra \R^{d_1}$
is a Lipschitz continuous map with $\lambda_\infty(0)=0$, we show that $\parallel x_n\parallel
\leq K^*(1+\parallel y_n\parallel)$ 
for some $K^*>0$. Subsequently, we analyze the slower recursion using the following ODE:
\begin{align}\label{gc} 
\dot{y}(t)=g_c(y(t)), \text{where } \mbox{ } g_c(y)=\frac{g(c\lambda_\infty(y),cy)}{c}.
\end{align}
We show that $\sup_n \parallel y_n\parallel <\infty$ with probability one, when the limiting 
ODE $\dot{y}(t)=g_\infty(y(t))$ (with $g_\infty(y)=\underset{c\ra\infty}{\lim}g_c(y)$) is stable 
to the origin. By making use of the fact that $\parallel x_n\parallel \leq K^*(1+\parallel y_n\parallel)$ 
(see above), 
we establish the stability of two timescale SA iterates given in \eqref{ttsaintro}.\\
\indent The paper is organized as follows. In section~\ref{oderes} we summarize certain results 
(mostly from \cite{SA,TTSAP}) regarding the scaled ODEs that will be used in our arguments. 
In sections~\ref{faster} and ~\ref{slower}, we present the analysis for the faster and slower 
timescale recursions, respectively. In section~\ref{example}, we apply our analysis to establish 
the stability of iterates in a two-timescale stochastic approximation algorithm arising in an 
application in reinforcement learning. Conclusions are provided in section~\ref{disc}. 
Finally, an Appendix at the end presents the Gronwall inequalities that are used in the proofs 
and a result whose proof follows along the lines of a similar result in \cite{SA}.

