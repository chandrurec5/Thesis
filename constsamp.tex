\section{Constraint sampling}
The most important work in the direction of constraint reduction is constraint sampling \cite{CS} wherein a reduced linear program (RLP) is solved instead of the ALP. While the objective function of the RLP is the same as that of the ALP, the RLP has only $m<<nd$ constraints sampled from the original ALP according to a given probability distribution. The reduced linear program is then given by
\small
\begin{align}\label{rlp}
\min_{J\in \R^n} &c^\top J\nn\\
\text{s.t}\mb &J(s)\geq g_a(s)+\alpha\sum_{s'}p_a(s,s')J(s'), \mb\forall (s,a) \in \I,
\end{align}
\normalsize
where $\I$ is the index set containing the $m$ sampled state-action pairs. Using the index set $\I$ we define an $nd\times m$ matrix $\M$, called the constraint sampling matrix as below.
\begin{definition}\label{csampmat}
Let $\I=\{(s_1,a_1),\ldots,(s_m,a_m)\}$ be the sampled state-action pairs and let $q_i\stackrel{\Delta}{=}s_i+(a_i-1)\times n,\forall i=1,\ldots,m$. Then the constraint sampling matrix associated with the index set $\I$ is given by
\begin{align}
\M(i,j)&=1, \mb\text{if}\mb q_i=j\nn\\
&=0, \mb\text{otherwise}.
\end{align}
\end{definition}
The RLP can then be represented in short as
\begin{align}\label{rlpshort}
\min_{r\in \N} &c^\top \Phi r\nn\\
\text{s.t}\mb & \M^\top E \Phi r \geq \M^\top H \Phi r,
\end{align}
where $\M$ is a constraint sampling matrix as in Definition~\ref{csampmat} and $\N\subset \R^k$ is a bounded set such that $\tr \in \N$. Note that the feasible set of the RLP is a superset of the feasible set of the ALP.\\
The RLP based on constraint sampling has been found to perform well empirically in application domains ranging from controlled queuing networks (see section $6.2$ of \cite{ALP} and section $7$ of \cite{SALP}) to Tetris (see \cite{CST} and section $6$ of \cite{SALP}). However, the theoretical guarantees for the RLP are available only under a restricted setting \cite{CS}. We present the main result of \cite{CS} on constraint sampling, and to that end define the following:
\begin{definition}\label{sampdist}
Given a policy $u$ and Lyapunov function $\psi$ in the column span of $\Phi$, we define probability distributions $\mu_u$, $\mu_{u,\psi}$ and constant $\theta$ as 
% Lyapunov function $\psi$ in the column span of $\Phi$, we define the probability distribution $\mu_{u,\pis}$ as
\begin{align}
\mu^\top&\stackrel{\Delta}{=}(1-\alpha)c^\top (I-\alpha P_u)^{-1},\nn\\
\mu_{u,\psi}(s,a)&\stackrel{\Delta}{=}\frac{\mu_{u}(s) \psi(s)}{\mu_u^\top \psi d},\nn\\
\theta&\stackrel{\Delta}{=}\frac{(1+\beta_{\psi})}{2}\frac{\mu_{u^*}^\top \psi}{c^\top J^*}\sup_{r\in \N}||J^*-\Phi r||_{\mn}.
\end{align}
\end{definition}
We now recall Theorem~$3.1$ of \cite{CS} below.
\begin{theorem}\label{csresult}
Let $\epsilon$ and $\delta$ be scalars in $(0,1)$. Let $u^*$ be the optimal policy and $\I$ an index set containing the $m$ state-action pairs sampled independently according to the distribution $\psi_{u^*,\psi}$, for some Lyapunov function $\psi$, where
\begin{align}
m\geq \frac{16d\theta}{(1-\alpha)\epsilon}\big(k\ln\frac{48d\theta}{(1-\alpha)\epsilon}+\ln\frac{2}{\delta}\big).
\end{align}
Let $\tilde{r}$ be an optimal	solution of the ALP and let $\hat{r}$ be the solution of the corresponding RLP, then with probability at least $1-\delta$, we have
\begin{align}
||J^*-\Phi \hat{r}||_{1,c}\leq ||J^*-\Phi \tilde{r}||_{1,c}+\epsilon||J^*||_{1,c}.
\end{align}
\end{theorem}
\textbf{Motivation for our work:}\\
The result in Theorem~\ref{csresult} has a significant limitation in that the sampling distribution $\mu_{u^*,\psi}$ requires the knowledge of $u^*$. The optimal policy $u^*$ might not be available in practice and hence it would not be possible to even formulate the specific RLP for which the bound in Theorem~\ref{csresult} applies. However, as mentioned before, the RLP has performed reasonably well even when the sampling distribution is not $\mu_{u^*,\psi}$. Thus there is a gap between the theoretical understanding of the RLP and its practical efficacy which merits our attention. The gap also indicates that the RLP might be a special case of a generalized constraint reduction scheme with provable performance guarantees. Understanding such a generalized method would result in an ALP based ADP techinque that would have theoretical performance guarantees while being practically useful. In particular, we wish to answer the following open questions.
\begin{itemize}
\item  As a natural generalization of the RLP, what happens if we define a generalized reduced linear program (GRLP) whose constraints are positive linear combinations of the original constraints of the ALP?
\item Unlike \cite{CS} which provides error bounds for a specific RLP formulated using an idealized sampling distribution, is it possible to provide error bounds for any GRLP (and hence any RLP)?
\end{itemize}
In this paper, we address both of the questions above.
