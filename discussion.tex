\FloatBarrier
\begin{table}[H]
\resizebox{\columnwidth}{!}{
\begin{tabular}{|c|c|c|}\hline
Choice of $W$&	$||J^*-\hj||_{1,c}$, with $\zeta=0.9$ &	$||J^*-\hj||_{1,c}$, with $\zeta=0.999$ \\\hline
$W_G$ \eqref{wdes} with $m=50$& $220$&		$82$\\\hline
RANDOM& $5.04\times 10^4$&	$1.25\times 10^7$\\\hline
\end{tabular}
}
\caption{Shows performance metrics for $Q_L$.}% Second row corresponds to $W$ as in \eqref{wdes} and the third row shows quantities averaged over $10$ random positive $4000\times 50$ matrices.}
\label{pref}
\end{table}
\section{Conclusion}
Solving MDPs with large number of states is of practical interest. However, when the number of states is large, it is difficult to calculate the exact value function. ALP is a widely studied ADP scheme that computes an approximate value function. Whilst the ALP offers sound theoretical guarantees it is also plagued by the presence of a large number of constraints. This calls for constraint reduction/approximation techniques. In this paper, we dealt with the GRLP and provided the error bounds, and by doing so we solved a major open problem of analytically justifying linear function approximation of the constraints. A salient feature of the error analysis was that it did not include any idealized assumptions. We demonstrated the effectiveness of the theory in a practical example of controlled queues. Future directions include providing more sophisticated error bounds based on Lyapunov functions, two-time scale actor-critic scheme to solve the GRLP, and basis function adaptation schemes to tune the $W$ matrix.
%\vspace{-10pt}
\begin{figure}[H]
\resizebox{\columnwidth}{!}{
\begin{tabular}{cc}
%\begin{tikzpicture}
\begin{axis}[scale=0.8, transpose legend,
legend columns=1,
legend style={at={(0.5,-0.1)},anchor=north},
]
\addplot[domain=-60:60,thick,smooth,variable=\y,black]  plot ({-74.000000+-1.000000*\y},{\y});
\addplot[domain=-60:60,dashed,variable=\y,black]  plot ({-124.000000+-2.000000*\y},{\y});
\addplot[mark=*,black] plot file{../nips_exp/r_ALP};
\addplot[domain=-60:60,thick,smooth,variable=\y,black]  plot ({-242.000000+-10.800000*\y},{\y});
\addplot[domain=-60:60,dashed,variable=\y,black]  plot ({-292.000000+-11.800000*\y},{\y});
\addplot[domain=-60:60,dashed,variable=\y,black]  plot ({-24.000000+9.800000*\y},{\y});
\addplot[domain=-60:60,dashed,variable=\y,black]  plot ({-192.000000+9.800000*\y},{\y});
\addplot[domain=-60:60,dashed,variable=\y,black]  plot ({-174.000000+-3.000000*\y},{\y});
\addplot[domain=-60:60,dashed,variable=\y,black]  plot ({-342.000000+-12.800000*\y},{\y});
\addplot[domain=-60:60,dashed,variable=\y,black]  plot ({-224.000000+-4.000000*\y},{\y});
\addplot[domain=-60:60,dashed,variable=\y,black]  plot ({-392.000000+-13.800000*\y},{\y});
\addplot[domain=-60:60,dashed,variable=\y,black]  plot ({-274.000000+-5.000000*\y},{\y});
\addplot[domain=-60:60,dashed,variable=\y,black]  plot ({-442.000000+-14.800000*\y},{\y});
\addplot[domain=-60:60,dashed,variable=\y,black]  plot ({-324.000000+-6.000000*\y},{\y});
\addplot[domain=-60:60,dashed,variable=\y,black]  plot ({-492.000000+-15.800000*\y},{\y});
\addplot[domain=-60:60,dashed,variable=\y,black]  plot ({-374.000000+-7.000000*\y},{\y});
\addplot[domain=-60:60,dashed,variable=\y,black]  plot ({-542.000000+-16.800000*\y},{\y});
\addplot[domain=-60:60,dashed,variable=\y,black]  plot ({-424.000000+-8.000000*\y},{\y});
\addplot[domain=-60:60,dashed,variable=\y,black]  plot ({-592.000000+-17.800000*\y},{\y});
\addplot[domain=-60:60,dashed,variable=\y,black]  plot ({-474.000000+-18.800000*\y},{\y});
\addplot[domain=-60:60,dashed,variable=\y,black]  plot ({-642.000000+-28.600000*\y},{\y});
\begin{comment}
\end{comment}
\addlegendentry{Active Constraints}
\addlegendentry{Inactive Constraints}
\addlegendentry{$\tr$}
\end{axis}
\end{tikzpicture}

%&
%\begin{tikzpicture}
\begin{axis}[scale=0.8,xmin=-500,xmax=400,transpose legend,
legend columns=1,
legend style={at={(0.5,-0.1)},anchor=north},
]
\addplot[domain=-60:60,smooth,variable=\y,black]  plot ({-133.000000+1.950000*\y},{\y});
\addplot[domain=-60:60,dashed,variable=\y,red]  plot ({-74.000000+-1.000000*\y},{\y});
\addplot[domain=-60:60,dotted,variable=\y,black]  plot ({-333.000000+-9.400000*\y},{\y});
\addplot[mark=*,green] plot file{../nips_exp/r_RLP};
\addplot[mark=o,blue] plot file{../nips_exp/r_ALP};
\addplot[domain=-60:60,smooth,variable=\y,black]  plot ({-233.000000+-7.400000*\y},{\y});
\addplot[domain=-60:60,dotted,variable=\y,black]  plot ({-433.000000+-11.400000*\y},{\y});
\addplot[domain=-60:60,dotted,variable=\y,black]  plot ({-533.000000+-18.300000*\y},{\y});
\addplot[domain=-60:60,dashed,variable=\y,red]  plot ({-242.000000+-10.800000*\y},{\y});
\addlegendentry{Active Constraints of GRLP}
\addlegendentry{Active Constraints of ALP}
\addlegendentry{Inactive Constraints of GRLP}
\addlegendentry{$\tj$}
\addlegendentry{$\hj$}
\end{axis}
\end{tikzpicture}

\begin{comment}
\begin{tikzpicture}
\begin{axis}[legend pos=north west, legend columns=2, scale=0.75]
\addplot[mark=square,black]  plot file {./nips_exp/gjb};
\addplot[mark=triangle,black]  plot file {./nips_exp/gtjb};
%\addplot[mark=*,black]  plot file {./nips_exp/V};
\addplot[mark=o,black]  plot file {./nips_exp/V_g};
%\addplot[dashed,black]  plot file {./nips_exp/jb};
\tiny
\addlegendentry{$\Gamma\bj$}
\addlegendentry{$\tg\bj$}
%\addlegendentry{$J^*$}
\addlegendentry{$\tv$}
%\addlegendentry{$\bj$}
\end{axis}
\end{tikzpicture}
\end{comment}
\begin{tikzpicture}
\begin{axis}[legend pos=south east,legend columns=2, scale=0.75,ymin=-100,xlabel=\Large State,ylabel=\Large Negative of value function]
\addplot[mark=square,red]  plot file {../nips_exp/V};
%\addplot[mark=triangle,black]  plot file {./nips_exp/V_g};
\addplot[mark=o,blue]  plot file {../nips_exp/V_alp};
\addplot[dashed,black]  plot file {../nips_exp/V_rlp};
%\addplot[dashed,mark=*,black]  plot file {./nips_exp/V_t};
\Large
\addlegendentry{$-J^*$}
%\addlegendentry{$-\tv$}
\addlegendentry{$-\tj$}
\addlegendentry{$-\hj$}
%\addlegendentry{$-\hv$}
\end{axis}
\end{tikzpicture}

\begin{comment}
\begin{tikzpicture}
\begin{axis}[legend pos=north west, legend columns=2, scale=0.75]
\addplot[mark=square,black]  plot file {./nips_exp/gjb};
\addplot[mark=triangle,black]  plot file {./nips_exp/gtjb};
%\addplot[mark=*,black]  plot file {./nips_exp/V};
\addplot[mark=o,black]  plot file {./nips_exp/V_g};
%\addplot[dashed,black]  plot file {./nips_exp/jb};
\tiny
\addlegendentry{$\Gamma\bj$}
\addlegendentry{$\tg\bj$}
%\addlegendentry{$J^*$}
\addlegendentry{$\tv$}
%\addlegendentry{$\bj$}
\end{axis}
\end{tikzpicture}
\end{comment}
\begin{tikzpicture}
\begin{axis}[legend pos=south east,legend columns=1, scale=0.75,ymin=-2000,xlabel=\Large State,ylabel=\Large{N}egative of value function]
\addplot[mark=.,black,blue]  plot file {../nips_exp_bkp/p9/Vl};
\addplot[mark=.,dashed,black]  plot file {../nips_exp_bkp/p9/Vl_rlp};
\addplot[mark=.,red]  plot file {../nips_exp_bkp/p999/Vl_rlp};
\addplot[mark=.,only marks]  plot file {../nips_exp_bkp/p999/Vl_rlp};
\Large
\addlegendentry{$-J^*$}
\addlegendentry{$-\hj, \zeta=0.9.$}
\addlegendentry{$-\hj, \zeta=0.999.$}
\end{axis}
\end{tikzpicture}

\end{tabular}
}
%\caption{Plots corresponding to $Q_S$. The dotted and solid lines in the left plot show the inactive and active constraints of the GRLP respectively, the dashed lines show the active constraints of the ALP. The right plot shows the \emph{negative} (thereby depicting positive cost instead of negative reward) of the various approximate value functions.}
\caption{Plot corresponding to $Q_S$ on the left and $Q_L$ on the right.}
\label{q1}
\end{figure}

%\vspace{-10pt}
\begin{comment}
\textbf{Lyapunov Functions:}
The error bounds are in terms of $||J^*-\Phi r^*||_\infty$ and $||\Gamma \bj-\tg\bj||_\infty$ and it can be argued that the basis functions might not approximate $J^*$ uniformly over all the states. This problem can be alleviated easily by making use of Lyapunov functions as in \cite{ALP} and showing that the operators $\Gamma$ and $\tg$ are contraction maps in a modified $L_\infty$ norm. %Since such a procedure requires additional algebra we have omitted its discussion in this paper and will provide the same in a longer version.\\
\end{comment}
\begin{comment}
\textbf{Reinforcement Learning:}
An important aspect of the GRLP is its amenability to the reinforcement learning (RL) \cite{Sutton} setting. In the RL setting, the model information is limited to the knowledge of the sample trajectories and solution can be obtained by a primal/dual gradient scheme that makes use of the Lagrangian function. The Lagrangian function corresponding to ALP and GRLP can be written as
\begin{align}\label{lag}
\tilde{L}(r,\lambda)=c^\top \Phi r+\lambda^\top (T\Phi r-\Phi r),\nn\\  \hat{L}(r,q)=c^\top \Phi r+q^\top W^\top (T\Phi r-\Phi r),
\end{align}
respectively. The additional insight from \eqref{lag} is that the GRLP can also be interpreted as linear function approximation of the Lagrangian multipliers, i.e., $\lambda\approx W q$. \cite{dolgov} takes such a view to formulate an approximate dual linear program (ADLP) as a dimensionality reduction technique. However, \cite{dolgov} provides neither an error analysis for the ADLP nor an RL algorithm. \cite{ALP} deals with an RL algorithm based on the ALP, nevertheless, the Lagrangian multipliers are approximated by a \emph{non-linear} function approximator.\\
\textbf{Unified view of various constraint approximation methods:}
An interesting direction is to extend the analytical machinery developed in this paper to constraint relaxation methods such as the smoothed approximate linear program (SALP) in \cite{SALP}. The key will be to appropriately modify the ALUB projection operator $\tg$ to suit the SALP. It will also be interesting to find the connection between the analytical arguments presented here based on contraction maps and the ones in \cite{CS} based on concentration inequalities.\\
\textbf{Basis Adaptation for W:} The paper justifies linear function approximation of the constraints by providing the error bounds. An interesting research direction is to find basis adaptation schemes that tune for $W$.
\end{comment}
\begin{comment}
\textbf{Empirical Evidence:}\\
It is well known that constraint sampling works in the case of large MDPs like Tetris \cite{CST}. Since the RLP is a special case of GRLP the results obtained in this paper further bolster the validity of such a solution approach for large MDPs.
\end{comment}
