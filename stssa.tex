\section{Single Timescale Stochastic Approximation}\label{stssa}
\subsection{Stability and Convergence}
We now state without proof, the convergence and stability results as in \cite{SA} for the single timescale SA scheme in \eqref{rmsa}. We assume the following on the quantities in \eqref{rmsa}:\\
\begin{assumption}\label{both}
\mbox{ }\\
\begin{enumerate}
\item The map $h \colon \mathbf{R}^d \rightarrow \mathbf{R}^d$ is Lipschitz: $||h(x)-h(y)|| \leq L||x-y||$ for some $0 <L<\infty$
\item Step-sizes $\{a(n)\}$ are positive scalars satisfying
\begin{align}
\sum_n a(n)=\infty,\sum_n a(n)^2<\infty.
\end{align}
\item $\{M_n\}$ is a martingale difference sequence with respect to the increasing family of $\sigma$-fields
\begin{align}
\mathcal{F}_n\stackrel{def}{=}\sigma(x_m,M_m,m\leq n)=\sigma(x_0,M_1,\ldots,M_n), n\geq 0. \text{That is}\nn\\
\mathbf{E}[M_{n+1}|\mathcal{F}_n]=0\mbox{ } a.s., n\geq 0.
\end{align}
Also, $\{M_n\}$ are square-integrable with
\begin{align}
\mathbf{E}[||M_{n+1}||^2|\mathcal{F}_n]\leq K(1+||x_n||^2) \mbox{ } a.s., n\geq0,
\end{align}
for some constant $K>0$.
\item \label{scaleode} The functions $h_c(x)\stackrel{def}{=}\frac{h(cx)}{c}, c\geq 1, x \in \mathbf{R}^d$, satisfy $h_c(x)\rightarrow h_\infty(x)$ as $c\rightarrow \infty$ uniformly on compacts for some $h_\infty \in C(\mathbf{R}^d)$. Furthermore, the ODE
\begin{align}\label{scalestab}
\dot{x}(t)=h_\infty(x(t)),
\end{align}
has the origin in $\R^d$ as its unique globally asymptotically stable equilibrium.
\end{enumerate}
\end{assumption}
\begin{theorem}\label{stssastab}
Under Assumption~\ref{both} we have $\sup_n ||x_n||<\infty$.
\end{theorem}
\begin{proof}
See proof of Theorem~$7$, Chapter~$3$ of \cite{SA}.
\end{proof}\\
We now define the timescale associated with \eqref{rmsa}.
\begin{definition}
Define timescale $t(n), n\geq0$, as $t(0)=0$, $t(n)=\sum^{n-1}_{m=0}a(m), n\geq0$. Note that $t(n)\uparrow \infty$ as $n\rightarrow \infty$.
\end{definition}
We now construct a interpolated trajectory $\bar{x}(t)$ using iterates $\{x_n\}, n\geq 0$ and $t(n)$ as below. 
\begin{align}\label{traj}
	\bar{x}(t(n))=&x_n\nn\\
	\bar{x}(t)=&x_n+(x_{n+1}-x_n)	\frac{t-t(n)}{t(n+1)-t(n)}, t \in [t(n), t(n+1)].
\end{align}
 $\bar{x}(t)$ is continuous in $t$ and serves as a $\emph{continuous-time}$ representation of the iterates $x_n$. Let $x^s(t)$ be the solution to the ODE in \eqref{odebasic}, for $t \geq s$, with $x^s(s)=\bar{x}(s)$. Our aim is to establish a connection between $\bar{x}(t)$ and $x^s(t)$ for $t\geq s$. It turns out that (Theorem~\ref{stssaconv}) $\bar{x}(t)$ $\emph{tracks}$ $x^s(t)$. The following convergence result (Theorem~\ref{stssaconv}) is a consequence of Assumption~\ref{both} and Theorem~\ref{stssastab}. 
\begin{theorem}\label{stssaconv}
Under Assumption~\ref{both}, for any $T>0$,
\begin{align}
\lim_{s\rightarrow \infty}\sup_{t\in[s,s+T]}||\bar{x}(t)-x^s(t)|| =0, a.s.
\end{align}
Also, almost surely, the sequence $\{x_n\}$ converges to a compact connected internally chain transitive set of \eqref{odebasic}. In particular, if ODE \eqref{odebasic} has an unique globally asymptotically stable equilibrium $x^*$, then the iterates of \eqref{rmsa} converge to $x^*$.
\end{theorem}
\begin{proof}
Proof follows form Theorem~\ref{stssastab}, Lemma~$1$ \& Theorem~$2$ of Chapter~$2$ of \cite{SA}.
\end{proof}\\
We hasten to point out that if the iterates $\{x_n\}$ evolve within a compact set then the result of Theorem~\ref{stssastab} holds trivially, i.e., $\sup_n||x_n||<\infty$. In such a scenario, condition $4$ of Assumption~\ref{both} can be dropped for the convergence result in Theorem~\ref{stssaconv}.\\
We now quickly review the basic idea behind establishing stability of the SA scheme \eqref{rmsa} as presented in \cite{SA}.
\subsection{Rough sketch of Stability Analysis}
The idea behind the stability analysis is to monitor the interpolated trajectory $\bar{x}(t)$ at time instants $\{T_n\}, n\geq 0$ which are roughly $T>0$ apart (i.e., $T_{n+1}-T_n\approx T$). If the trajectory $\bar{x}(t)$ goes out of the unit ball around the origin, it is rescaled to the unit ball. The rescaled trajectory is denoted by $\hat{x}(t)$ and for $t \in [T_n,T_{n+1}]$ $\hat{x}(t)=\frac{\bar{x}(t)}{||\bar{x}(T_n)||}$. Since $\hat{x}(t)$ is always within the unit ball ($\sup_n||\hat{x}(t)|| <\infty$), convergence result (i.e., Theorem~\ref{stssaconv}) can be applied to it. Due to condition $4$ of Assumption~\ref{both}, we know that if $\bar{x}(t)$ drifts to infinity then $\hat{x}(t)$ will track the limiting ODE in \eqref{scalestab}. Since \eqref{scalestab} has origin as its unique asymptotically stable equilibrium the rescaled trajectory has to drift towards origin. Since the original trajectory and the rescaled trajectory differ by a scaling factor, $\bar{x}(t)$ cannot drift to infinity.
\begin{comment}
Application of stability result (Theorem~\ref{stssastab}) for the recursion in \eqref{couprec} is not straightforward due to the presence of two timescales $t^f$ and $t^s$. It is not clear whether one needs to rescale with respect to one of the timescales (faster/slower) or both. 
%It turns out that we have to scale the interpolated trajectory with respect to both timescales. 
For instance, if we scale with respect to the faster timescale, we cannot immediately conclude the boundedness of $x_n$ because $y_n$ couples through $h(x_n,y_n)$ and we do not know about the boundedness of $y_n$. To add to the difficulty $y_n$ evolve along a slower timescale. In this paper, we tackle these issues by,
\begin{enumerate}
\item \textbf{Step 1:} First studying the interpolated trajectories with respect to the faster timescale and rescale them every $T^f_n$ time instants in the faster timescale, to show that the faster timescale iterates are bounded with a ball whose radius is proportional to the slower timescale iterates.
\item \textbf{Step 2:} Then studying the interpolated trajectories with respect to the slower timescale and rescale them every $T^s_n$ time instants in the slower timescale. Use result from \textbf{Step 1} to establish the boundedness of the slower timescale iterates and the boundedness of the faster timescale iterates follows automatically.
\end{enumerate}
We also make appropriate assumptions on the scaled ODEs.
\end{comment}
